

% This is a simple sample document.  For more complicated documents take a look in the exercise tab. Note that everything that comes after a % symbol is treated as comment and ignored when the code is compiled.

\documentclass[UTF8]{ctexart}
\usepackage{amsmath}
\usepackage{geometry}
\usepackage{lipsum}
\usepackage{sectsty}
\usepackage{graphicx}
\raggedbottom
\sectionfont{\bfseries\Large\raggedright}
\geometry{left=2.5cm,right=2.5cm,top=2.5cm,bottom=2.5cm}

\date{May 8, 2022} % Sets date for date compiled

% The preamble ends with the command \begin{document}
\begin{document} % All begin commands must be paired with an end command somewhere
    \title{%
    Machine Learning for Used Car Price Prediction \\
    \large COMP 562 Final Project}

    \author{Peter Yao, Xuan Bai, James He, Ziqian Zhao}

    \maketitle
    
    \section{Introduction} % creates a section
    The used car market is a significant sector of the US economy, 
    with 40.9 million cars sold and a value of \$196 billion in 2021 [2][3]. 
    A wide set of features determine the price of a used car, including age, mileage, and model [7].
    Several models have used used in the past to predict the value of used cars based on these features, 
    including machine learning tools like XGBoost, random forest, and lasso regression [3][4]. 
    We evaluated the effectiveness of several machine learning models on predicting the value of used cars
    Three regression models were selected and trained on a dataset of used car transactions on eBay. 
    The accuracy and speed of the models was then tested, and one of them was selected to be
    used as the basis of a fourth ensembling model that we developed. 
    The accuracy, speed, and feature importance of the four models was then compared and analyzed.
        \newline

    \section{Dataset and Cleaning}
        \subsection{Dataset}
        Our dataset was a tabular dataset sourced from Kaggle and consists of approximately 160,000 sales 
        records of US and Canadian used cars on eBay over a 20 month period from 2019 to 2020. The dataset 
        contains 13 columns which list ID (a unique value assigned to each transaction), body type (sedan, SUV, etc), 
        number of cylinders, drive type (RWD, 4WD, etc), price sold, year (the year the car was manufactured), zipcode, mileage, make, model, 
        year sold, trim, and engine. \newline
        \subsection{Data Cleaning}
        Not all features in our dataset were useful, and many entries contained null values or errors. 
        4 features deemed largely irrelevant or which had too many errors and null values were dropped from the dataset

        We excluded rows with Canadian zipcodes from the dataset, and as several zipcodes were missing the last
        2 digits, we dropped the last 2 digits from all zipcodes. The first three digits of a zipcode, while 
        not as specific as the full zipcode, are still able to provide meaningful geographic information about
        the location of the car's sale.
        
        We considered dropping the feature for number of cylinders as well, as many gas-powered cars were listed as 
        having 0 cylinders in what were likely errors, but this would cause all electric vehicles to be excluded 
        from the dataset as well. Instead, cars with 0 cylinders were dropped from the dataset unless they were
        Teslas, as the number of non-Tesla electric vehicles in the dataset was very small.

        Rows with outlier values in any of the columns (these often contained typos) were dropped, as well as
        any remaining rows with null values. In the end, about 60,000 of the original ~160,000 entries remained.
        \newline
        \subsection{Data Visualization}
        The result of visual data is shown below:
        \newline
    
    \section{Methodology}
    After preprocessing the data, we trained three models – HistGradientBoostingRegression, RandomForest, 
    and LinearRegression – on the training dataset. The performance of each model was evaluated on the validation dataset 
    in terms of accuracy and speed. The evaluation was carried out using mean squared error (MSE) and mean absolute error (MAE). 
    We then selected one of the three (HistGradientBoostingRegressor) based on the evaluation results and used an ensembling technique to enhance its effectiveness. 
    
    Furthermore, we also compared the results of using one-hot encoding vs. ordinal encoding on the three models, and the ensemble model.
    \newline

    
    \section{Models and Results}
        \subsection{Model Results}
        Chart depicting mean absolute error, mean squared error, time to fit, and time to run for each model over 3 trials as the chart shown below:

        
        On each trial, a new model was instantiated and trained. For RandomForests, the random state was changed from 0 to 42 to 101 on trials 1, 2, and 3. This is to prevent the model from training the same way on each trial. For Linreg, the model would be the same for each trial so MSE and MAE results are only shown for trial 1
        
        According to the result, the HistGradientBoostingRegressor with an ensembling technique (Ensemble HG) has the lowest mean absolute error and mean squared error compare to other models, which indicates that Ensemble HG is the best-performing model in terms of prediction accuracy. In terms of computational efficiency, HistGradientBoostingRegressor (HGBReg) takes less time to fit and run, it will be a good alternative if looking for a balance between accuracy and computational efficiency.
        \newline
        \subsection{Model Results on Feature Importance}
    
         Based on the chart, for Hreg and Rfreg models, the permutation importance follows the order: Year $>$ Mileage $>$ NumCylinders $>$ Model $>$ Make $>$ BodyType $>$ DriveType $>$ zipcode. And in the coefficients of Linear regression, Year, Make, and zipcode have a positive impact, and Mileage, Model, BodyType, and DriveType have a negative impact. Overall, Year and Mileage have the most permutation importance for predicting the target variable in all three models. \newline


    \section{Discussion}
    Our project used different models to predict used car prices based on the dataset obtained from Kaggle. However, this dataset was sourced from eBay and had a relatively small number of records. Getting a bigger and more diverse range of data may produce more robust results.
    \newline
    Furthermore, we probably can explore more models, like recurrent neural networks (RNN) and knearest neighbors algorithm (KNN), in the future. Both of them can be potential good models for used car price prediction.
    \newline
    Also, we can attempt to do some hyperparameter tuning and look for optimal hyperparameters.
    \newline
    
    \section{References}
    \begin{enumerate}
        \item Ts. (2020, November 16). US used car sales data. Kaggle. \newline
        \url{https://www.kaggle.com/datas/ets/tsaustin/us-used-car-sales-data}

        \item Moore, C. J. (2022, January 18). U.S. used-vehicle sales record set in 2021. Automotive News. \url{https://www.autonews.com/retail/used-car-sales-set-us-record-2021-cox-automotive-say}

        \item US used car market size & share analysis - industry research report - growth trends. US Used Car Market Size & Share Analysis - Industry Research Report - Growth Trends. (n.d.). \newline
        \url{https://www.mordorintelligence.com/industry-reports/united-states-used-car-market}

        \item Used cars price prediction and valuation using data mining techniques. (2021).\newline
        \url{https://scholarworks.rit.edu/cgi/viewcontent.cgi?article=12220&context=theses}

        \item Jain, S. (2021, May 17). Used car price prediction using supervised machine learning. Medium. \url{https://shubh17121996.medium.com/used-car-price-prediction-using-supervised-machine-learning-ea9dace76686}

        \item Hagerty, M. (2022, January 27). Factors that can affect used car trade-in value. Capital One Auto Navigator. \url{https://www.capitalone.com/cars/learn/managing-your-money-wise/factors-that-can-affect-used-car-tradein-value/1224}

        \item Gong, J., Peng, L., & Li, J. (2018a). A study on the factors affecting the value of used cars in Panzhihua region. Proceedings of the 2nd International Forum on Management, Education and Information Technology Application (IFMEITA 2017). \url{https://doi.org/10.2991/ifmeita17.2018}.
    \end{enumerate}
\end{document} % This is the end of the document
